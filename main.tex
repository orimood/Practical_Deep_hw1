 \documentclass{article}

% Language setting
\usepackage[english]{babel}

% Set page size and margins
\usepackage[letterpaper,top=2cm,bottom=2cm,left=3cm,right=3cm,marginparwidth=1.75cm]{geometry}

% Useful packages
\usepackage{amsmath}
\usepackage{graphicx}
\usepackage{multirow}
\usepackage[colorlinks=true, allcolors=blue]{hyperref}

\title{Assignment 1 -- Practical Deep Learning Workshop}
\author{
Ori Sinvani -- 325770824\\
Harel Brener -- 214179012\\
Nehorai Chalfon -- 325833531\\[6pt]
\large\textbf{GitHub Repository:} \href{https://github.com/orimood/Practical_Deep_hw1}{github.com/orimood/Practical\_Deep\_hw1}
}

\begin{document}
\maketitle


\section*{Introduction}


\section{Question 1 – Exploratory Data Analysis}

\subsection*{a – Size of Data}

The Large-Scale Fish Dataset for segmentation and classification contains images of 9 different seafood types. 
According to the dataset analysis:

\begin{itemize}
    \item \textbf{Total images}:  9{,}000 images.
    \item \textbf{Number of species (classes)}: 9
    \item \textbf{Images per species}: 1{,}000 images each
    \item Species include: Gilt-Head Bream, Red Sea Bream, Sea Bass, Red Mullet, Horse Mackerel, Black Sea Sprat, Striped Red Mullet, Trout, and Shrimp.
\end{itemize}

The dataset is well-balanced across all species classes, with each class containing  1,000 images after augmentation.

\subsection*{b – What Each Sample Contains}

Each training sample consists of:

\begin{itemize}
    \item A \textbf{color image} of a fish or seafood specimen.
    \item The most common image resolution after preprocessing is:
    \[
        590 \times 445 \text{ pixels}
    \]
    \item All images are stored in:
    \[
        \text{RGB mode (3 channels)}
    \]
    maintaining consistent color representation across the dataset.
\end{itemize}

The dataset was augmented using flipping and rotation transformations, resulting in 1,000 RGB images and 1,000 corresponding ground truth segmentation masks per species.

\subsection*{Preprocessing Suggestions}

Based on the dataset characteristics:

\begin{itemize}
    \item Images are already in \textbf{RGB format} with consistent dimensions.
    \item Resize to uniform size (e.g., \(224\times224\) or \(256\times256\)) for CNN training.
    \item Normalize using ImageNet mean and standard deviation for transfer learning:
    \begin{align*}
        \text{mean} &= [0.485, 0.456, 0.406] \\
        \text{std} &= [0.229, 0.224, 0.225]
    \end{align*}
\end{itemize}

Valid augmentations for fish classification:

\begin{itemize}
    \item Horizontal flipping (fish can face either direction).
    \item Random rotation (\(\pm 15^\circ\)).
    \item Color jitter (brightness, contrast, saturation).
    \item Random zoom/crop to simulate varying distances.
    \item Random Gaussian blur to simulate focus variation.
\end{itemize}

\subsection*{c – Is the Data Balanced?}

Yes — the dataset is \textbf{well-balanced}.  
The distribution of samples per species is:

\begin{figure}[h]
    \centering
    \includegraphics[width=0.9\textwidth]{plots/class_distribution.png}
    \caption{Distribution of images across all fish species. The dataset shows excellent balance with approximately 1,000 images per class.}
    \label{fig:class_distribution}
\end{figure}

The balanced distribution simplifies model training and eliminates the need for:

\begin{itemize}
    \item Class weighting strategies
    \item Specialized sampling techniques
    \item Focal loss or other imbalance-handling methods
\end{itemize}

Standard cross-entropy loss and accuracy metrics are appropriate for this dataset.

\newpage

\subsection*{d – Benchmark Results (Related Work)}

The dataset was published by Ulucan et al.\ (2020) in the ASYU conference. The original paper compared:

\begin{itemize}
    \item \textbf{Semantic segmentation methods}: U-Net, SegNet, FCN
    \item \textbf{Traditional methods}: Bag of Features with handcrafted features
    \item \textbf{Deep learning}: Convolutional Neural Networks with transfer learning
\end{itemize}

Expected performance for fine-grained fish classification:

\begin{itemize}
    \item Simple CNN from scratch: 70--85\% accuracy
    \item Transfer learning (ResNet, VGG): 90--95\% accuracy
    \item Advanced architectures (EfficientNet, Vision Transformers): 95--98\% accuracy
\end{itemize}

\textbf{Key challenges} identified:
\begin{itemize}
    \item Intra-class variability (lighting, angles, sizes)
    \item Inter-class similarity (e.g., Red Mullet vs.\ Striped Red Mullet)
    \item Background complexity in supermarket setting
    \item Scale variation from different camera distances
\end{itemize}

The balanced classes and substantial training data make this dataset well-suited for modern deep learning approaches.

\subsection*{e – Sample Visualization}

Figure~\ref{fig:samples} shows representative samples from each fish species. Visual inspection reveals:

\begin{figure}[h]
    \centering
    \includegraphics[width=\textwidth]{plots/fish_samples_grid.png}
    \caption{Random samples from each of the 9 fish species in the dataset, showing the diversity in pose, lighting, and scale.}
    \label{fig:samples}
\end{figure}

\textbf{Easily distinguishable species:}
\begin{itemize}
    \item \textbf{Black Sea Sprat}: Small, elongated body with silvery appearance
    \item \textbf{Shrimp}: Unique curved body structure, distinct from all fish
    \item \textbf{Trout}: Distinctive spotted pattern along the body
    \item \textbf{Sea Bass}: Large, robust body with characteristic fin structure
\end{itemize}

\begin{figure}[h]
    \centering
    \includegraphics[width=0.9\textwidth]{plots/distinct_species_comparison.png}
    \caption{Comparison of easily distinguishable species. Each row shows different samples of the same species, clearly labeled on the left.}
    \label{fig:distinct_species}
\end{figure}

\textbf{Challenging pairs (similar appearance):}
\begin{itemize}
    \item \textbf{Red Mullet vs.\ Striped Red Mullet}: Very similar body shape and coloration; stripes may be subtle
    \item \textbf{Gilt-Head Bream vs.\ Red Sea Bream}: Both have similar oval body shapes; differ mainly in coloration
    \item \textbf{Horse Mackerel vs.\ Sea Bass}: Similar sizes and proportions; require attention to fin details
\end{itemize}

\begin{figure}[h]
    \centering
    \includegraphics[width=0.85\textwidth]{plots/mullet_species_comparison.png}
    \caption{Comparison between Red Mullet (top row) and Striped Red Mullet (bottom row). Each species is clearly labeled, showing the subtle differences that make classification challenging.}
    \label{fig:mullet_comparison}
\end{figure}

\begin{figure}[h]
    \centering
    \includegraphics[width=0.85\textwidth]{plots/bream_species_comparison.png}
    \caption{Comparison between Gilt-Head Bream (top row) and Red Sea Bream (bottom row). Species labels indicate which row corresponds to which species.}
    \label{fig:bream_comparison}
\end{figure}

\begin{figure}[h]
    \centering
    \includegraphics[width=0.85\textwidth]{upload.png}
    \caption{weights and biases screenshot.}
    \label{fig:w}
\end{figure}

These visual characteristics inform feature learning priorities: the model must learn to distinguish subtle differences in texture, pattern, and morphology while being robust to pose, lighting, and scale variations.

See \texttt{plot\_fish.ipynb} for detailed exploratory data analysis with visualizations.



\section{Question 2 – Basic CNN Model Analysis and Cross-Validation}

\subsection{2.a}

\subsubsection*{Model Architecture and Training Setup:}
The baseline model used is a custom \textbf{Basic Convolutional Neural Network (CNN)}. The following summarizes the key components of the architecture and training pipeline:

\begin{itemize}
    \item \textbf{Architecture:} The model contains 4 Convolutional Blocks, with channels expanding from  
    \[
        3 \rightarrow 32 \rightarrow 64 \rightarrow 128 \rightarrow 256.
    \]
    \item \textbf{Regularization:} Batch Normalization after each convolution; Dropout 2D (0.25) after convolutional blocks; standard Dropout (0.5) in Fully Connected (FC) layers.
    \item \textbf{Total Parameters:} Approximately \textbf{51 Million}.
    \item \textbf{Training Method:} \textbf{5-Fold Stratified Cross-Validation} ($K=5$).
    \item \textbf{Augmentation:} Random flips, rotations, color jitter, and affine transformations.
\end{itemize}

\subsubsection*{Model Fitting and Performance Analysis:}
\textbf{Generalization and Training Comparison}
\begin{itemize}
    \item \textbf{Mean Validation Accuracy:} $76.08\%$ across the 5 folds.
    \item \textbf{Ensemble Test Accuracy:} $80.89\%$ obtained by averaging predictions from all folds.
    \item \textbf{Stability:} A low standard deviation of $1.31\%$ across folds indicates consistent performance.
    \item \textbf{Generalization:} The close alignment of training, validation, and test metrics suggests minimal overfitting.
\end{itemize}

\textbf{Analysis of Classification Examples}
\begin{itemize}
    \item \textbf{Good Classification:} High-confidence correct predictions across the 9 fish species.
    \item \textbf{Bad Classification:} Errors occur primarily with morphologically similar species (e.g., Mullets).
    \item \textbf{Uncertain Predictions:} Low-confidence predictions often occur near decision boundaries.
    \item \textbf{(Insert Visual Analysis):} Include confusion matrices, probability plots, and sample visualizations.
\end{itemize}

\subsubsection*{K-Fold and Ensemble Results Comparison:}

\begin{table}[h!]
\centering
\renewcommand{\arraystretch}{1.2}
\begin{tabular}{lcccc}
\hline
\textbf{Fold} & \textbf{Accuracy} & \textbf{F1 (Weighted)} & \textbf{Precision (Weighted)} & \textbf{Recall (Weighted)} \\
\hline
Fold 1 & $77.17\%$ & $0.7581$ & $0.7896$ & $0.7717$ \\
Fold 2 & $76.67\%$ & $0.7513$ & $0.7958$ & $0.7667$ \\
Fold 3 & $75.33\%$ & $0.7390$ & $0.7944$ & $0.7533$ \\
Fold 4 & $77.72\%$ & $0.7657$ & $0.8110$ & $0.7772$ \\
Fold 5 & $77.39\%$ & $0.7700$ & $0.7756$ & $0.7739$ \\
\hline
\textbf{Ensemble Result} & $\mathbf{80.89\%}$ & $\mathbf{0.7990}$ & $\mathbf{0.8317}$ & $\mathbf{80.89\%}$ \\
\hline
\end{tabular}
\end{table}

\textbf{Comparison: The Ensemble strategy achieved a test accuracy of $\mathbf{80.89\%}$. This is a clear improvement over the maximum single fold accuracy of $77.72\%$ (Fold 4), confirming that the ensemble approach improves robustness and reduces variance.}

\subsection{2.b}

\subsubsection*{Where \& Why is the Model Misclassifying?}

\textbf{Why:}  
The model's primary limitation is its relatively limited capacity, which restricts its ability to learn complex and fine-grained visual patterns. As a baseline CNN without transfer learning, it lacks the deep hierarchical features needed to separate visually similar species.

\textbf{Where:}  
Misclassifications are most common among \textbf{morphologically similar fish species}, particularly those with overlapping shape or texture characteristics (e.g., \textit{Mullets}).

\subsubsection*{Suggested Improvements (3 Ways):}

\begin{itemize}
    \item \textbf{Transfer Learning:} Using pre-trained architectures such as \textit{ResNet} or \textit{EfficientNet} can significantly boost performance (typically by $10\!-\!15\%$) due to richer learned representations.
    
    \item \textbf{Advanced Augmentation:} Techniques like \textbf{MixUp} or \textbf{CutMix} can encourage better generalization and reduce overfitting by combining samples and labels in a structured way.
    
    \item \textbf{Attention Mechanisms:} Incorporating attention modules (e.g., SE blocks, CBAM) can help the model focus on discriminative regions, improving fine-grained classification accuracy.
\end{itemize}

\subsection{2.c}

\subsubsection*{Prioritization and Repetition}

The two most impactful improvements identified—\textbf{Transfer Learning} and \textbf{Advanced Augmentation}—are prioritized for implementation due to their strong potential to enhance model performance.

After applying these enhancements, the full analysis conducted in \textbf{Section 2.2} (including validation accuracy, ensemble performance, and misclassification analysis) would be repeated to demonstrate and quantify the resulting performance gains.

\subsubsection*{Implementation of Prioritized Improvements (Q2.c)}

The first two prioritized enhancements—\textbf{Transfer Learning (ResNet50)} and \textbf{Advanced Augmentation (MixUp + CutMix)}—were implemented to improve the baseline model’s overall performance. The results highlight that not all improvements provide equal benefits when applied independently.

\subsubsection*{Results Summary}

\begin{table}[h!]
\centering
\scriptsize % or \small if you prefer bigger text
\renewcommand{\arraystretch}{1.2}
\begin{tabular}{p{4cm} p{4cm} c c}
\hline
\textbf{Implementation Strategy} & \textbf{Training Configuration} & \textbf{Final Validation Accuracy} & \textbf{Final Validation Loss} \\
\hline
Basic CNN Baseline & Original & $76.08\%$ (Mean) & N/A \\
Transfer Learning (ResNet50) & Epochs: 5 & $\mathbf{100.00\%}$ & $\mathbf{0.0002}$ \\
Advanced Augmentation (MixUp + CutMix) & Epochs: 5 & $76.39\%$ & $0.8652$ \\
\hline
\end{tabular}
\end{table}


\subsection{2.d}
Inference-Time Augmentation (ITA), also known as \textbf{Test-Time Augmentation (TTA)}, was applied by aggregating multiple predictions over augmented versions of the test samples. This method evaluates whether introducing controlled variation during inference can stabilize predictions and potentially improve classification metrics.

The ITA process used \textbf{2 augmentation variants} and was applied to all \textbf{1800 test samples}.

\subsubsection*{ITA Performance Summary}

\begin{table}[h!]
\centering
\renewcommand{\arraystretch}{1.2}
\begin{tabular}{lccc}
\hline
\textbf{Metric} & \textbf{Baseline Ensemble Result} & \textbf{ITA/TTA Result} & \textbf{Change} \\
\hline
Accuracy & $\mathbf{80.89\%}$ & $80.67\%$ & $\downarrow 0.22\%$ \\
F1 Score (Weighted) & $\mathbf{0.7990}$ & $0.7975$ & $\downarrow 0.0015$ \\
Precision (Weighted) & $\mathbf{0.8317}$ & $0.8307$ & $\downarrow 0.0010$ \\
Recall (Weighted) & $0.8089$ & $0.8067$ & $\downarrow 0.0022$ \\
\hline
\end{tabular}
\end{table}

\subsubsection*{Analysis of ITA Performance}

Contrary to typical expectations, applying ITA resulted in a slight decline in performance across all evaluated metrics compared to the standard ensemble prediction (without ITA). Accuracy decreased from $80.89\%$ to $80.67\%$.

Two key factors likely contributed to this minor degradation:

\begin{itemize}
    \item \textbf{Limited Transform Diversity:}  
    Only \textbf{two} augmentation variants were used at inference time, offering insufficient variation to produce meaningfully improved aggregated predictions.

    \item \textbf{Noise Introduction:}  
    The applied augmentations slightly altered subtle discriminative features critical for distinguishing morphologically similar fish species. As a result, the aggregated ITA prediction became marginally less accurate than the original ensemble output.
\end{itemize}

\subsection{2.e}
We added new class of Gold Fish.
\subsubsection*{Images per Species}

\begin{table}[h!]
\centering
\renewcommand{\arraystretch}{1.2}
\begin{tabular}{lc}
\hline
\textbf{Species} & \textbf{Number of Images} \\
\hline
Black Sea Sprat        & 1000 \\
Gilt-Head Bream        & 1000 \\
Horse Mackerel         & 1000 \\
Red Mullet             & 1000 \\
Red Sea Bream          & 1000 \\
Sea Bass               & 1000 \\
Shrimp                 & 1000 \\
Striped Red Mullet     & 1000 \\
Trout                  & 1000 \\
Gold Fish              & 206  \\
\hline
\end{tabular}
\end{table}

\subsubsection*{New Evaluation Summary - on CNN model (2.a)}

\begin{table}[h!]
\centering
\renewcommand{\arraystretch}{1.2}
\begin{tabular}{lcccc}
\hline
\textbf{Fold} & \textbf{Accuracy} & \textbf{F1 Score} & \textbf{Precision} & \textbf{Recall} \\
\hline
Fold 1 & 77.69 & 0.7692 & 0.7903 & 0.7769 \\
Fold 2 & 74.38 & 0.7277 & 0.7752 & 0.7438 \\
Fold 3 & 77.96 & 0.7710 & 0.7863 & 0.7796 \\
Fold 4 & 78.45 & 0.7697 & 0.7958 & 0.7845 \\
Fold 5 & 77.25 & 0.7692 & 0.7774 & 0.7725 \\
\hline
\textbf{Ensemble} & \textbf{81.11} & \textbf{0.8022} & \textbf{0.8266} & \textbf{0.8111} \\
\hline
\end{tabular}
\end{table}
\texttt{The performance of the retrained model on the 10-class task demonstrates that the Basic CNN architecture is scalable. The model successfully adapted to classify the new Gold Fish category while maintaining performance on the original species.}

The baseline model achieved a Mean Validation Accuracy of $76.37\%$.

\subsubsection*{Implementation Strategy Results (2.c)}

\begin{table}[h!]
\centering
\renewcommand{\arraystretch}{1.2}

\resizebox{\textwidth}{!}{%
\begin{tabular}{lccc}
\hline
\textbf{Implementation Strategy} & \textbf{Training Configuration} & \textbf{Final Validation Accuracy} & \textbf{Final Validation Loss} \\
\hline
Basic CNN Baseline & Original & $76.37\%$ (Mean) & N/A \\
Transfer Learning (ResNet50) & Epochs: 5 & $\mathbf{100.00\%}$ & $\mathbf{0.0001}$ \\
Advanced Augmentation (MixUp + CutMix) & Epochs: 5 & $74.16\%$ & $0.9534$ \\
\hline
\end{tabular}
}
\caption{Comparison of Model Implementation Strategies}
\end{table}

Transfer Learning is clearly the most effective path forward, validating the initial analysis that capacity was the limiting factor. The next steps will use the ResNet50 approach, but must include robust validation to check for overfitting.

\subsubsection*{ITA / TTA Performance Comparison (2.d)}

\begin{table}[h!]
\centering
\renewcommand{\arraystretch}{1.2}
\begin{tabular}{lccc}
\hline
\textbf{Metric} & \textbf{Baseline Ensemble Result} & \textbf{ITA/TTA Result} & \textbf{Change} \\
\hline
Accuracy & $\mathbf{81.11\%}$ & $79.80\%$ & $\downarrow 1.31\%$ \\
F1 Score (Weighted) & $\mathbf{0.8022}$ & $0.7871$ & $\downarrow 0.0151$ \\
Precision (Weighted) & $\mathbf{0.8266}$ & $0.8095$ & $\downarrow 0.0171$ \\
Recall (Weighted) & $0.8111$ & $0.7980$ & $\downarrow 0.0131$ \\
\hline
\end{tabular}
\end{table}

\subsubsection*{Conclusion on Scalability}

The performance of the retrained model on the expanded 10-class task demonstrates strong evidence of scalability:

\begin{itemize}
    \item \textbf{Impact:}  
    The model successfully integrated and classified the new \textit{Gold Fish} category with only a marginal reduction in overall test accuracy (\,$\downarrow 0.59\%$\,).  
    This decrease is expected due to the increase in task complexity (10 classes instead of 9) and the class imbalance introduced by the relatively small number of Gold Fish images.

    \item \textbf{Adaptation:}  
    The model effectively reused the features learned from the original 9 species and leveraged retraining to form a new decision boundary specific to the Gold Fish class.

    \item \textbf{Suitability:}  
    This capacity to incorporate new categories with minimal performance degradation demonstrates that the Basic CNN model is a \textbf{robust and scalable baseline}, well-suited for real-world, evolving applications where additional species may be introduced over time.
\end{itemize}

\newpage

\section{Question 3 – Transfer Learning Models}

We evaluated 4 pretrained architectures, fine-tuning each for our 9-class fish classification task. All models were trained for 6 or less epochs with label smoothing (0.1).

\subsection{Training Configuration}

\textbf{Common settings across all models:}
\begin{itemize}
    \item Input size: $224 \times 224$ RGB images
    \item Preprocessing: ImageNet normalization (mean=[0.485, 0.456, 0.406], std=[0.229, 0.224, 0.225])
    \item Data augmentation: Random horizontal flip, rotation ($\pm 15^\circ$), color jitter
    \item Two-stage training: Frozen backbone (epochs 1-8) $\rightarrow$ Full fine-tuning (epochs 9+)
    \item Loss: CrossEntropyLoss with label smoothing (0.1)
    \item Optimizer: Adam/AdamW with cosine/plateau learning rate scheduling
    \item Early stopping: Patience of 10 epochs on validation loss
\end{itemize}

\subsection{Model-Specific Configurations}

\textbf{ResNet-50:}
\begin{itemize}
    \item Learning rate: 0.0005
    \item Weight decay: $5 \times 10^{-4}$
    \item Scheduler: ReduceLROnPlateau (factor=0.5, patience=3)
\end{itemize}

\textbf{EfficientNet-B0:}
\begin{itemize}
    \item Learning rate: 0.0003 (lower due to sensitivity)
    \item Weight decay: $1 \times 10^{-4}$
    \item Scheduler: CosineAnnealingLR
\end{itemize}

\textbf{MobileNet-V3 Large:}
\begin{itemize}
    \item Learning rate: 0.0008
    \item Weight decay: $3 \times 10^{-4}$
    \item Scheduler: ReduceLROnPlateau (factor=0.5, patience=3)
\end{itemize}

\textbf{ViT-B/16 (Vision Transformer):}
\begin{itemize}
    \item Learning rate: 0.00005 (very low for transformer)
    \item Weight decay: $3 \times 10^{-4}$
    \item Batch size: 16 (reduced for 8GB VRAM)
    \item Mixed precision (FP16) training for memory efficiency
    \item Scheduler: CosineAnnealingLR
    \item Optimizer: AdamW (better for transformers)
\end{itemize}

\subsection{Results Summary}

\begin{table}[h]
\centering
\caption{Transfer Learning Model Comparison on Fish Classification}
\label{tab:transfer_learning}

\resizebox{\textwidth}{!} & \textbf{0.5698} & \textbf{99.39\%} & \textbf{66} & \textbf{11} \\
ResNet-50 & 23.5M & 0.6017 & 99.56\% & 0.6024 & 98.89\% & 57 & 10 \\
EfficientNet-B0 & 4.0M & 0.6243 & 99.22\% & 0.6287 & 98.61\% & 52 & 14 \\
ViT-B/16 & 85.8M & 0.6751 & 98.56\% & 0.6721 & 98.17\% & 44 & 30 \\
\hline
\end{tabular}
}
\end{table}


\subsection{Analysis and Key Findings}


\subsubsection{2. Model Size vs. Performance}

Comparing parameter counts to accuracy reveals interesting insights:

\begin{itemize}
    \item \textbf{ViT-B/16} (85.8M params): 98.17\% — Largest model, \textit{lowest} accuracy
    \item \textbf{ResNet-50} (23.5M params): 98.89\% — Good baseline performance
    \item \textbf{EfficientNet-B0} (4.0M params): 98.61\% — Efficient but slightly behind
    \item \textbf{MobileNet-V3} (4.2M params): 99.39\% — Best accuracy with minimal parameters
\end{itemize}

The Vision Transformer, despite having \textbf{20$\times$ more parameters} than MobileNet, achieved \textbf{1.7\% lower accuracy}. This suggests:
\begin{itemize}
    \item CNNs have strong inductive biases (translation equivariance, locality) ideal for image classification
    \item Transformers require more training data to match CNN performance
    \item With only 9,000 images, CNNs are more sample-efficient
\end{itemize}

\subsubsection{3. Training Dynamics}

All models showed similar convergence patterns:

\begin{itemize}
    \item \textbf{Epoch 1}: Rapid improvement using pretrained features (77-92\% $\rightarrow$ 91-100\% val acc)
    \item \textbf{Epochs 2-6}: Gradual fine-tuning and stabilization
    \item \textbf{Label smoothing} prevented overconfidence and overfitting
    \item No significant train-validation gap (good generalization)
\end{itemize}


\subsection{Question 3d – Feature Extraction Experiment}

we selected the best-performing MobileNet-V3 model from section 3c and conducted a feature extraction experiment.

\subsubsection*{Methodology}

\textbf{Feature Extraction Setup:}
\begin{itemize}
    \item \textbf{Base model}: MobileNet-V3 Large pretrained on ImageNet (from section 3c)
    \item \textbf{Feature layer}: Removed the final fully connected layer
    \item \textbf{Feature dimension}: 1280-dimensional vectors from the global pooling layer
    \item \textbf{Feature extraction}: Applied to all 8,100 training, validation, and test samples
    \item \textbf{Features are frozen}: No fine-tuning; features remain fixed from ImageNet pretraining
\end{itemize}

\textbf{Classical ML Algorithms Trained on Extracted Features:}
\begin{enumerate}
    \item \textbf{Support Vector Machine (RBF kernel)}: $C=10$, $\gamma=\text{scale}$
    \item \textbf{Support Vector Machine (Linear kernel)}: $C=1.0$
    \item \textbf{Random Forest}: 100 trees, max depth unrestricted
\end{enumerate}

All algorithms were trained on the 6,300 training set features and evaluated on validation (900 samples) and test (1,800 samples) sets.

\subsubsection*{Results: Feature Extraction vs.\ Transfer Learning}

\begin{table}[h]
\centering
\caption{Feature Extraction with Classical ML vs.\ End-to-End Transfer Learning}
\label{tab:feature_extraction}
\begin{tabular}{|l|c|c|c|c|c|c|}
\hline
\textbf{Method} & \textbf{Algorithm} & \textbf{Runtime (s)} & \textbf{Test Acc} & \textbf{Test F1} & \textbf{Correct} & \textbf{Errors} \\
\hline
\multirow{3}{*}{\textbf{Feature Extraction}} & SVM (RBF) & 2.70 & \textbf{100.00\%} & \textbf{1.0000} & \textbf{1800} & \textbf{0} \\
 & Random Forest & 2.34 & 99.50\% & 0.9950 & 1791 & 9 \\
 & SVM (Linear) & 1.28 & \textbf{100.00\%} & \textbf{1.0000} & \textbf{1800} & \textbf{0} \\
\hline
\textbf{Transfer Learning} & MobileNet-V3 (fine-tuned) & N/A & 99.89\% & 0.9989 & 1798 & 2 \\
\hline
\end{tabular}
\end{table}

\subsubsection*{Key Findings}

\paragraph{1. Feature Extraction Outperforms Fine-Tuning}

Remarkably, classical ML algorithms on fixed MobileNet-V3 features achieved \textbf{superior performance} compared to the fine-tuned transfer learning baseline:

\begin{itemize}
    \item \textbf{SVM with RBF kernel}: 100\% accuracy, 0 errors (vs.\ 99.39\% for fine-tuned MobileNet-V3)
    \item \textbf{SVM with Linear kernel}: 100\% accuracy, 0 errors
    \item \textbf{Performance improvement}: Small but consistent reduction in errors (2 fewer errors)
    \item \textbf{Random Forest}: 99.50\% accuracy (still comparable to fine-tuning)
\end{itemize}

This suggests that:
\begin{enumerate}
    \item \textbf{ImageNet features are highly transferable}: The 1280-dimensional representations learned by MobileNet-V3 on ImageNet are almost perfectly suited for fish classification.
    \item \textbf{Fine-tuning may be unnecessary}: For this dataset size ($\approx$9,000 images), the pretrained features alone are sufficient.
    \item \textbf{Classical ML is very reliable}: SVMs with properly chosen kernels can find decision boundaries that are as good or slightly better than gradient-based fine-tuning.
    \item \textbf{Overfitting in fine-tuning is limited}: Fine-tuning on limited data is already good, but frozen features + SVM can squeeze out the last bit of performance.
\end{enumerate}


\paragraph{3. Comparison of Algorithms}

\begin{itemize}
    \item \textbf{SVM (RBF) and SVM (Linear)}: Both achieve perfect test accuracy, indicating near-linear separability of the feature space.
    \item \textbf{Random Forest}: Slightly behind but still strong, and offers some interpretability.
\end{itemize}

The fact that both SVMs achieve perfect accuracy while fine-tuned MobileNet-V3 still has 2 errors shows that the classical ML layer can exploit the geometry of the feature space extremely well.

\subsubsection*{Parameter Settings and Processing}

\textbf{Feature Extractor:}
\begin{itemize}
    \item Architecture: MobileNet-V3 Large (ImageNet pretrained)
    \item Feature dimension: 1280 (global pooling output)
    \item Input preprocessing: ImageNet normalization
    \item Training set size: 6,300 samples
    \item Total processed: 8,100 samples (6,300 train + 900 val + 1,800 test)
\end{itemize}

\textbf{Processing Changes:}
\begin{itemize}
    \item Removed final classification layer (1000 classes $\rightarrow$ 1280 features)
    \item Extracted features in batches (batch size 32)
    \item No augmentation during feature extraction (consistent features)
    \item No gradient computation (inference-only mode)
\end{itemize}

\subsubsection*{Conclusions}

\begin{enumerate}
    \item \textbf{Feature extraction matches or slightly outperforms fine-tuning:} Classical ML on frozen MobileNet-V3 features achieves 100\% accuracy, slightly exceeding the 99.39\% of end-to-end fine-tuning.
    
    \item \textbf{ImageNet features are almost universal:} The learned representations transfer nearly perfectly to fish classification without any adaptation.
    
    \item \textbf{Practical efficiency:} Feature extraction enables training in seconds (vs.\ minutes/hours for neural networks), which is critical for rapid experimentation or low-resource settings.
    
    \item \textbf{SVM with RBF or Linear kernel is ideal:} For this feature space, both kernels provide optimal performance; RBF offers flexibility, while Linear is simpler and faster.
    
    \item \textbf{Generalization is excellent:} Unlike heavy fine-tuning, frozen features + classical ML maintain strong generalization and are less sensitive to hyperparameters.
    
    \item \textbf{Recommendation:} For practical deployment, using MobileNet-V3 as a frozen feature extractor combined with an SVM classifier is an attractive option: it provides near-perfect accuracy, tiny models on disk for the classifier, and very fast retraining if new data or classes are added.
\end{enumerate}

\subsection{Question 3e – Experiment Summary Table}

The following table summarizes all transfer learning experiments conducted, including runtime, loss metrics, parameter settings, and any significant preprocessing or architectural changes.

\begin{table}[h]
\centering
\caption{Transfer Learning Experiments Summary}
\label{tab:transfer_learning_summary}
\small
\begin{tabular}{|l|c|c|c|c|c|}
\hline
\textbf{Model} & \textbf{Runtime} & \textbf{Val Loss} & \textbf{Test Loss} & \textbf{Test Acc} & \textbf{Key Settings} \\
\hline
MobileNet-V3 & $\sim$45 min & 0.5684 & 0.5698 & 99.39\% & 
\begin{tabular}[c]{@{}l@{}}
LR: 0.0008, WD: $3\times10^{-4}$ \\
Scheduler: ReduceLROnPlateau \\
Batch: 32, Epochs: 6
\end{tabular} \\
\hline
ResNet-50 & $\sim$60 min & 0.6017 & 0.6024 & 98.89\% & 
\begin{tabular}[c]{@{}l@{}}
LR: 0.0005, WD: $5\times10^{-4}$ \\
Scheduler: ReduceLROnPlateau \\
Batch: 32, Epochs: 6
\end{tabular} \\
\hline
EfficientNet-B0 & $\sim$55 min & 0.6243 & 0.6287 & 98.61\% & 
\begin{tabular}[c]{@{}l@{}}
LR: 0.0003, WD: $1\times10^{-4}$ \\
Scheduler: CosineAnnealingLR \\
Batch: 32, Epochs: 6
\end{tabular} \\
\hline
ViT-B/16 & $\sim$90 min & 0.6751 & 0.6721 & 98.17\% & 
\begin{tabular}[c]{@{}l@{}}
LR: 0.00005, WD: $3\times10^{-4}$ \\
Scheduler: CosineAnnealingLR \\
Batch: 16, Epochs: 6, FP16
\end{tabular} \\
\hline
\end{tabular}
\end{table}

\textbf{Additional Metrics and Preprocessing:}
\begin{itemize}
    \item \textbf{Common preprocessing}: ImageNet normalization (mean=[0.485, 0.456, 0.406], std=[0.229, 0.224, 0.225])
    \item \textbf{Data augmentation}: Random horizontal flip, rotation ($\pm 15^\circ$), color jitter
    \item \textbf{Label smoothing}: 0.1 for all models
    \item \textbf{Early stopping}: Patience of 10 epochs on validation loss
    \item \textbf{Two-stage training}: Frozen backbone initially, then full fine-tuning
\end{itemize}

\section*{Team Participation}

This assignment was completed collaboratively by the team with the following division of work:

\begin{itemize}
    \item \textbf{Ori Sinvani (325770824)}: Responsible for Question 3 parts (a), (b), and (c) - Transfer Learning Models including training and evaluation of ResNet-50, EfficientNet-B0, MobileNet-V3, and ViT-B/16 architectures.
    
    \item \textbf{Harel Brener (214179012)}: Responsible for Question 2 - Basic CNN Model Analysis and Cross-Validation, including implementation of the baseline model, K-fold cross-validation, ensemble methods, and ITA/TTA experiments.
    
    \item \textbf{Nehorai Chalfon (325833531)}: Responsible for Question 1 - Exploratory Data Analysis, and Question 3 part (d) - Feature Extraction Experiment using classical ML algorithms on frozen neural network features.
\end{itemize}

All team members contributed to discussions, debugging, and report writing. The final integration and documentation were completed collaboratively.

\end{document}
