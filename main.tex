\documentclass{article}

% Language setting
% Replace `english' with e.g. `spanish' to change the document language
\usepackage[english]{babel}

% Set page size and margins
% Replace `letterpaper' with `a4paper' for UK/EU standard size
\usepackage[letterpaper,top=2cm,bottom=2cm,left=3cm,right=3cm,marginparwidth=1.75cm]{geometry}

% Useful packages
\usepackage{amsmath}
\usepackage{graphicx}
\usepackage[colorlinks=true, allcolors=blue]{hyperref}

\title{Assignment 1 – Practical Deep Learning Workshop}
\author{Ori Sinvani - 325770824
Harel Brener - 214179012}

\begin{document}
\maketitle

\begin{abstract}
Your abstract.
\end{abstract}

\section*{Introduction}


\section{Question 1 – Exploratory Data Analysis}

\subsection*{a – Size of Data}

The Large-Scale Fish Dataset for segmentation and classification contains images of 9 different seafood types collected from a supermarket in Izmir, Turkey.  
According to the dataset analysis:

\begin{itemize}
    \item \textbf{Total images}:  \(\approx 9{,}000\) images.
    \item \textbf{Number of species (classes)}: 9
    \item \textbf{Images per species}: \(\approx 1{,}000\) images each
    \item Species include: Gilt-Head Bream, Red Sea Bream, Sea Bass, Red Mullet, Horse Mackerel, Black Sea Sprat, Striped Red Mullet, Trout, and Shrimp.
\end{itemize}

The dataset is well-balanced across all species classes, with each class containing approximately 1,000 images after augmentation.

\subsection*{b – What Each Sample Contains}

Each training sample consists of:

\begin{itemize}
    \item A \textbf{color image} of a fish or seafood specimen.
    \item Images were collected using two cameras: Kodak Easyshare Z650 and Samsung ST60.
    \item The most common image resolution after preprocessing is:
    \[
        590 \times 445 \text{ pixels}
    \]
    \item All images are stored in:
    \[
        \text{RGB mode (3 channels)}
    \]
    maintaining consistent color representation across the dataset.
\end{itemize}

The dataset was augmented using flipping and rotation transformations, resulting in 1,000 RGB images and 1,000 corresponding ground truth segmentation masks per species.

\subsection*{Preprocessing Suggestions}

Based on the dataset characteristics:

\begin{itemize}
    \item Images are already in \textbf{RGB format} with consistent dimensions.
    \item Resize to uniform size (e.g., \(224\times224\) or \(256\times256\)) for CNN training.
    \item Normalize using ImageNet mean and standard deviation for transfer learning:
    \begin{align*}
        \text{mean} &= [0.485, 0.456, 0.406] \\
        \text{std} &= [0.229, 0.224, 0.225]
    \end{align*}
\end{itemize}

Valid augmentations for fish classification:

\begin{itemize}
    \item Horizontal flipping (fish can face either direction).
    \item Random rotation (\(\pm 15^\circ\)).
    \item Color jitter (brightness, contrast, saturation).
    \item Random zoom/crop to simulate varying distances.
    \item Random Gaussian blur to simulate focus variation.
    \item \textbf{Note}: Vertical flipping is \textbf{not valid} (fish do not swim upside down).
\end{itemize}

\subsection*{c – Is the Data Balanced?}

Yes — the dataset is \textbf{well-balanced}.  
The distribution of samples per species is:

\begin{itemize}
    \item \textbf{Images per species}: \(\approx 1{,}000\) for each class
    \item \textbf{Minimum}: 1{,}000 images
    \item \textbf{Maximum}: 1{,}000 images
    \item \textbf{Standard deviation}: Very low (\(<10\%\) of mean)
\end{itemize}

\begin{figure}[h]
    \centering
    \includegraphics[width=0.9\textwidth]{plots/class_distribution.png}
    \caption{Distribution of images across all fish species. The dataset shows excellent balance with approximately 1,000 images per class.}
    \label{fig:class_distribution}
\end{figure}

The balanced distribution simplifies model training and eliminates the need for:

\begin{itemize}
    \item Class weighting strategies
    \item Specialized sampling techniques
    \item Focal loss or other imbalance-handling methods
\end{itemize}

Standard cross-entropy loss and accuracy metrics are appropriate for this dataset.

\subsection*{d – Benchmark Results (Related Work)}

The dataset was published by Ulucan et al.\ (2020) in the ASYU conference. The original paper compared:

\begin{itemize}
    \item \textbf{Semantic segmentation methods}: U-Net, SegNet, FCN
    \item \textbf{Traditional methods}: Bag of Features with handcrafted features
    \item \textbf{Deep learning}: Convolutional Neural Networks with transfer learning
\end{itemize}

Expected performance for fine-grained fish classification:

\begin{itemize}
    \item Simple CNN from scratch: 70--85\% accuracy
    \item Transfer learning (ResNet, VGG): 90--95\% accuracy
    \item Advanced architectures (EfficientNet, Vision Transformers): 95--98\% accuracy
\end{itemize}

\textbf{Key challenges} identified:
\begin{itemize}
    \item Intra-class variability (lighting, angles, sizes)
    \item Inter-class similarity (e.g., Red Mullet vs.\ Striped Red Mullet)
    \item Background complexity in supermarket setting
    \item Scale variation from different camera distances
\end{itemize}

The balanced classes and substantial training data make this dataset well-suited for modern deep learning approaches.

\subsection*{e – Sample Visualization}

Figure~\ref{fig:samples} shows representative samples from each fish species. Visual inspection reveals:

\begin{figure}[h]
    \centering
    \includegraphics[width=\textwidth]{plots/fish_samples_grid.png}
    \caption{Random samples from each of the 9 fish species in the dataset, showing the diversity in pose, lighting, and scale.}
    \label{fig:samples}
\end{figure}

\textbf{Easily distinguishable species:}
\begin{itemize}
    \item \textbf{Black Sea Sprat}: Small, elongated body with silvery appearance
    \item \textbf{Shrimp}: Unique curved body structure, distinct from all fish
    \item \textbf{Trout}: Distinctive spotted pattern along the body
    \item \textbf{Sea Bass}: Large, robust body with characteristic fin structure
\end{itemize}

\begin{figure}[h]
    \centering
    \includegraphics[width=0.9\textwidth]{plots/distinct_species_comparison.png}
    \caption{Comparison of easily distinguishable species. Each row shows different samples of the same species, clearly labeled on the left.}
    \label{fig:distinct_species}
\end{figure}

\textbf{Challenging pairs (similar appearance):}
\begin{itemize}
    \item \textbf{Red Mullet vs.\ Striped Red Mullet}: Very similar body shape and coloration; stripes may be subtle
    \item \textbf{Gilt-Head Bream vs.\ Red Sea Bream}: Both have similar oval body shapes; differ mainly in coloration
    \item \textbf{Horse Mackerel vs.\ Sea Bass}: Similar sizes and proportions; require attention to fin details
\end{itemize}

\begin{figure}[h]
    \centering
    \includegraphics[width=0.85\textwidth]{plots/mullet_species_comparison.png}
    \caption{Comparison between Red Mullet (top row) and Striped Red Mullet (bottom row). Each species is clearly labeled, showing the subtle differences that make classification challenging.}
    \label{fig:mullet_comparison}
\end{figure}

\begin{figure}[h]
    \centering
    \includegraphics[width=0.85\textwidth]{plots/bream_species_comparison.png}
    \caption{Comparison between Gilt-Head Bream (top row) and Red Sea Bream (bottom row). Species labels indicate which row corresponds to which species.}
    \label{fig:bream_comparison}
\end{figure}

These visual characteristics inform feature learning priorities: the model must learn to distinguish subtle differences in texture, pattern, and morphology while being robust to pose, lighting, and scale variations.

See \texttt{plot\_fish.ipynb} for detailed exploratory data analysis with visualizations.




\end{document}
